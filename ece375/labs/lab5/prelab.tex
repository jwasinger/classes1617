% template created by: Russell Haering. arr. Joseph Crop
\documentclass[12pt,letterpaper]{article}
\usepackage{anysize}
\marginsize{2cm}{2cm}{1cm}{1cm}

\begin{document}

\begin{titlepage}
    \vspace*{4cm}
    \begin{flushright}
    {\huge
        ECE 375 Lab 5 - Prelab\\[1cm]
    }
    {\large
        Large Number Arithmetic
    }
    \end{flushright}
    \begin{flushleft}
    Lab Time: Wednesday 5-7
    \end{flushleft}
    \begin{flushright}
    Jared Wasinger
    \vfill
    \rule{5in}{.5mm}\\
    TA Signature
    \end{flushright}

\end{titlepage}

\section{Prelab}
\begin{enumerate}
  \item \textbf{For this lab, you will be asked to perform arithmetic operations on numbers that are larger than 8 bits. To be successful at this, you will need to understand and utilize many of the various arithmetic operations supported by the AVR 8-bit instruction set. List and describe all of the addition, subtraction, and multiplication instructions (i.e. ADC, SUBI, FMUL, etc.) available in AVR’s 8-bit instruction set.}\\
  \begin{itemize}
    \item ADD Rd, Rr : $Rd \leftarrow Rd + Rr$ - add two registers
    \item ADC Rd, Rr : $Rd \leftarrow Rd + Rr + c$ - add two registers with carry
    \item SUB Rd, Rr : $Rd \leftarrow Rd - Rr $ - subtract two registers
    \item SUBC Rd, Rr: $Rd \leftarrow Rd - Rr - c$ - subtract two registers with carry
    \item SUBI Rd, K : $Rd \leftarrow Rd - K$ - subtract immediate value from register
    \item ADIW Rd+1:Rd, K : $Rd+1:Rd \leftarrow Rd+1:Rd + K$ - Add immediate to word
    \item SBIW Rd+1:Rd, K : $Rd+1:Rd \leftarrow Rd+1:rd - K$ - Subtract immediate from word
    \item MUL Rd, Rr      : $R1:R0 \leftarrow Rd x Rr$ - Multiply unsigned
    \item MULS Rd, Rr     : $R1:R0 \leftarrow Rd x Rr$ - Multiply signed
    \item MULSU Rd, Rr    : $R1:R0 \leftarrow rd x Rr$ - Multiply signed with unsigned
    \end{itemize}

  \item \textbf{Write pseudocode for an 8-bit AVR function that will take two 16-bit numbers (from data memory addresses \$0111:\$0110 and \$0121:\$0120), add them together, and then store the 16-bit result (in data memory addresses \$0101:\$0100). (Note: The syntax "\$0111:\$0110" is meant to specify that the function will expect little-endian data, where the highest byte of a multi-byte value is stored in the highest address of its range of addresses.)}\\
    \begin{verbatim}
LD          tmp, $0110
LD          tmp2, $0120
ADD         tmp, tmp2
ST          low(result), tmp
ldi         tmp, $00
BRBC        carry_flag, add_higher
inc         tmp
add_higher:
LD         tmp2, $0111
ADD         tmp, tmp2
LD         tmp2, $0121
ADD         tmp, tmp2
ST          high(result), tmp
    \end{verbatim}


  \item \textbf{Write pseudocode for an 8-bit AVR function that will take the 16-bit number in \$0111:\$0110, subtract it from the 16-bit number in \$0121:\$0120, and then store the 16-bit result into \$0101:\$0100.}
    \begin{verbatim}
.def sourceAddrA=$1110
.def sourceAddrB=$0120

...
LD  tmp,  low(sourceAddrB)
LD  tmp2, low(sourceAddrA)
SUB tmp2, tmp
ST low(targetAddr), tmp2
LD tmp, high(sourceAddrB)
LD tmp2, high(sourceAddrA)
SUBC tmp2, tmp
ST high(targetAddr), tmp2
    \end{verbatim}



  \end{enumerate}
\end{document}
