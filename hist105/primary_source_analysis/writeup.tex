\documentclass{article}

\begin{document}
  \title{ HIST 105: Primary Source Analysis}
  \author{Jared Wasinger}
  
  \maketitle

  \begin{enumerate}
    \item The introduction to this primary source tells you about the author.  What facets of this author's identity are important for understanding the text? Why are these identity markers important? What complicates our understanding of authorship for this text?
      
      The author was a daughter of a high-ranking official in the Chinese government.  The text can be argued to espouse cultural values of the wealthy elite in Chinese society of the Tang dynasty.  This is evidenced by the author's discouragement of women participating in society outside of managing household affairs.  For example, the text insists that women "not peer over the outer wall or go beyond the outer courtyard" (page 9).  The assumption that a family should possess a large house with courtyard and surrounding wall is another way in which the life and social position of the author biases the text.

      Although the text has a decidedly conservative tone (by modern standards), it is impossible to draw conclusions about the true thoughts and beliefs of the author.  Due to the position of the author within the Tang government, combined with the lack of any teachings on the subject from Confucius himself, it is most likely that the true voice of the text is the Chinese Tang ruling class.
      

    \item \textbf{Intended Audience}
      \begin{enumerate}
      \item Who is the intended audience for this text? (you may want to consider markers like gender, class, location, etc.)



      \item How do you know the answer? Bring evidence (quotes or appropriate paraphrases) to prove it.
      \item How does the fact that this text was written for this specific audence influence how you understand this text? What does this tell us about the social context in which this text is written?

        The advice from the text is probably most applicable to middle-class households.  the author's view that women should keep a low-profile in society outside of the immediate family is shown by the supposed ideal behavior for a woman:

        "When walking, do not turn your head; when talking, do not open your mouth wide; ... when angry, do not raise your voice;" (page 9).

        The author's insistence that women not leave the confines of the household, coupled with strict rules of etiquette implies that the target audience is wealthy enough that women do not need to work outside the house to support their family economically.  For the lower class of society such as subsistance farmers, this would not be economically feasible.

        Despite being destined for a life in the household, women in the target audience are clearly expected to actively participate in home and family matters.  According to the text, women are tasked with many things including making shoes, producing silk from silk worms, weaving hemp, among other tasks (page 9).
      \end{enumerate}
    \item \textbf{Social Context} What can other aspects of this text tell us about the contours of the society which produced it, or assumptions that people living in this society made about their world?
      \begin{enumerate}
      \item After making your argument in a topic sentence, offer at least 2 quotes or kinds of quotes that tell us something specific about the scoiety.
      \item Then analyze them, i.e. explain  how each quote (or kind of quote) demonstrates something specific about the society.
      \end{enumerate}

      This text indicates that Chinese culture of the period valued ritual in many aspects of life.  One of the key indicators of this is the elaborate (and strict), process by which a woman should treat with guests in their household: "... walk slowly to the door and with lowered voice, invite her in.  Ask after her health and how her family is doing" (Page 10).
      The prescribed way in which women are supposed to serve their parents in law further supports the argument for the importance of ritual in Tang society:  "Respectfully serve your father-in-law.  Do not look at him directly, do not follow him around, and do not engage him in conversation.  If he has an order for you, listen and obey."
  \end{enumerate}
\end{document}
