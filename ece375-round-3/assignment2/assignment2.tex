\documentclass{article}
\usepackage{graphicx}
\usepackage{listings}
\usepackage{flexisym}
\usepackage{graphicx}
\usepackage{enumitem}

\begin{document}

  \title{ECE 375: Assignment 2}
  \author{Jared Wasinger}

  \maketitle

  \begin{enumerate}
    \item \textbf{Fetch Cycle} \begin{enumerate}
      \item $MAR \leftarrow PC, PC \leftarrow PC + 1$
      \item $MDR \leftarrow M(MAR)$
      \item $TEMP_{15:8} \leftarrow MDR, MAR \leftarrow PC$
      \item $MDR \leftarrow M(MAR), PC \leftarrow PC + 1$
      \item $TEMP_{7:0} \leftarrow MDR, MAR \leftarrow PC$
      \item $IR \leftarrow TEMP, MDR \leftarrow M(MAR), PC \leftarrow PC + 1$
      \item $TEMP_{15:8} \leftarrow MDR, MAR \leftarrow PC$
      \item $MDR \leftarrow M(MAR)$
      \item $TEMP_{7:0} \leftarrow MDR$
      \end{enumerate}
    \textbf{Execute Cycle} \begin{enumerate}
      \item $MAR \leftarrow SP, MDR \leftarrow TEMP_{7:0}$
      \item $M(MAR) \leftarrow MDR, SP \leftarrow SP + 1$
      \item $MAR \leftarrow SP, MDR \leftarrow TEMP_{15:8}$
      \item $MDR \leftarrow M(MAR)$
      \item $PC \leftarrow TEMP$
      \end{enumerate}
    \item adiw ZH:ZL, 32\\
      \textbf{Fetch Cycle}\begin{enumerate}
        \item $MAR \leftarrow PC$
        \item $MDR \leftarrow M(MAR), PC \leftarrow PC + 1$ 
        \item $IR \leftarrow MDR$
        \item $MAR \leftarrow PC$
        \item $MDR \leftarrow M(MAR)$
        \end{enumerate}
      \textbf{execute Cycle}\begin{enumerate}
        \item $AC \leftarrow MDR, MAR \leftarrow ZL$ ; AC gets immediate value
        \item $AC \leftarrow AC + r30$ ; add ZL to the immediate value
        \item $r30 \leftarrow AC$
        \item if Carry set:\begin{enumerate}
          \item $AC \leftarrow r31$
          \item $AC \leftarrow AC + 1$
          \item $r31 \leftarrow AC$
          \end{enumerate}
        \end{enumerate}
    \item\begin{enumerate}
      \item\begin{verbatim}
2. INIT: clr zero ; Set zero register to zero
3. MAIN: ldi  low(X), low(addrA) ; Load X low byte
4. ldi high(X), high(addrA); Load X high byte
5. ldi low(Z), low(LAddrP) ; Load Z low byte
6. ldi high(Z), high(LAddrP); Load Z high byte
7. ldi oloop, 2 ; Load counter
8. MUL16_OLOOP: ldi low(Y), low(addrB); Load low byte
9. ldi high(Y), high(addrB); Load high byte
        \end{verbatim}
      \item addrA - \$0203, addrB - \$010C
      \item 
      \end{enumerate}
    \item\begin{verbatim}
      0000            1100  kkkk  kkkk  kkkk
      ...
      0002            1101  kkkk  kkkk  kkkk
      0003            1001  0101  0001  1000
      ...
      000B            1110  KKKK  dddd  KKKK
      000C            1110  KKKK  dddd  KKKK
      000D            1110  KKKK  dddd  KKKK
      000E            1110  KKKK  dddd  KKKK
      000F            1100  kkkk  kkkk  kkkk
      ...
      100F            1011  0AAd  dddd  AAAA
      1010            1001  001r  rrrr  1001
      1011            1001  010d  dddd  0011
      1012            1001  001r  rrrr  1101
      1013            1001  0101  0000  1000
      \end{verbatim}
      \begin{itemize}
        \item 0000 - jump forward to \$0046, current address is \$0000, $\$0000 - \$0047 = -\$0047 = 0000 0100 0111$
        \item 0002 -
        \item 000B - KKKK KKKK is the value high(CTR) = high(\$0100) = (\$01) = 0000 0001, dddd is the target register (XH) = 27 = 0b11011 (first 1 implied) = 1011
        \item 000C - KKKK KKKK = low(CTR) = low(\$0100) = \$00 = 0000 0000, dddd is the target register (XL) = 0b11010 (first 1 implied) = 1010
        \item 000D - KKKK KKKK is the value high(DATA) = high(\$0101) = \$01 = 0000 0001, dddd is the target register = YH = r29 = 1101
        \item 000E - KKKK KKKK is the value low(DATA) = low(\$0101) = \$01 = 0000 0001, dddd is the target register YL = r28 = 1100
        \item 000F -
        \item 100F - d dddd is the target register = r16 = 10000, AA AAAA is the source address in I/O Space = PINA = \$19 = 11001 = 01 1001
        \item 1010 - r rrrr is the source register = r16 = 10000
        \item 1011 - d dddd is the destination register = r17 = 10001
        \item 1012 - r rrrr is the source register = r17 = 10001
      \end{itemize}
  \item\textbf{Source Code:}
    \begin{verbatim}
.include "m128def.inc"				; Include definition file

.def minimum=r16
.def count=r17
.def location=r18
.def mpr=r19

.equ COUNT_MAX = 7

.cseg
.ORG $0000
  rjmp    INIT

.ORG $0046
INIT:
  ;initialize stack
  ldi	mpr, low(RAMEND)
  out	SPL, mpr
  ldi	mpr, high(RAMEND)
  out	SPH, mpr

  rcall Min

Min:	ldi		XL, low(AddrNums)
		ldi		XH, high(AddrNums)
		ld		location, X
		ldi		count, COUNT_MAX
		mov		minimum, location

Loop:	dec		count
		inc		XL
		ld		location, X
		cp		minimum, location 
		brlo	EndLoop
		mov		minimum, location ; set a new maximum if we have found a higher number
EndLoop:
		cpi		count, 0
		brne	Loop
		ldi		XL, 0x08
		ldi		XH, 0x00
		st		X, minimum
		ret
    \end{verbatim}
  \end{enumerate}
\end{document}
